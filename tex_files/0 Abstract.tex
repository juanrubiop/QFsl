
%1.Talks about general application or general field of research
%2. Explains the challenge that is not answered yet
%3. Describe idea for tackling that challenge e.g. overall idea
%4 Methodology undertaken in research  e.g. tools, methods, steps taken.
%5. Main achievement of research supported by numerical results
%6. comparison with similar research in literature (qualitatively or quantitatively), and/or explain possible future applications of outcomes
The field of Quantum Computation is plagued by issues that limit the implementation and development of quantum systems and quantum algorithms. Issues which force the development of Hybrid Quantum-Classical algorithms, such as the quantum DisCoCat implementation for Natural Language Processing. These require a high processing cost and are susceptible to errors due to Out of Vocabulary words. 

In this work, we develop a framework to implement Few Shot Learning for Quantum Natural Language Processing, by modifying the encoding \mya and dividing it into two parts, the first one leveraging the vast corpus of classical training already available, and the second variationally training on the task. This framework is then put to the test to explore its behaviour and its power in extracting as much useful work from each call to a quantum system as possible.