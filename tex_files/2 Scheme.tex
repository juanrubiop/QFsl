\chapter{An Outline} \label{Scheme}

The contribution of this project to the field of QNLP is the development of the framework of Few Shot Learning to be used in running DisCoCat circuits and the testing to show its usefulness. This framework allows for the reduction in training resources at a marginal cost in accuracy that decreases with increasingly bigger circuits.

To set this up, we will first set the minimum theoretical background in chapter \ref{chap:Theory}, where the fields of Quantum Machine Learning, and Quantum Natural Language Processing will be covered, along with a description of DisCoCat and Few Shot Learning.

Chapter \ref{chap:Problem} will describe the specific problem we are trying to solve, along with the motivations for choosing and solving that specific problem.

Following an explanation of the framework in chapter \ref{chap:framework}, chapter \ref{chap:experiment} will detail the experimental design to test how this framework compares to traditional methods as well as the results of this testing. 

Finally, in chapter \ref{chap:conclusion} we will summarise what the experiments tell us about the framework, as well as its limitations, and what possible experimental avenues could be further explored.

Further information can be found in the appendices, including more information on bias in ML, the notion of quantum supremacy, and other interesting areas in QML which could benefit from FSL. All the code is linked to in the appendix as well.